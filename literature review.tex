
\documentclass[12pt]{article}
\usepackage[top=1cm, bottom=1cm, left=1cm, right=1cm]{geometry}
\begin{document}
	\title{A LITERATURE REVIEW ON CLOUD STORAGE SYSTEMS.}
	\author{Kikomeko musa Reg no.:15/u/6675/ps Student no.:215004259}
	\maketitle
	
	\section{CLOUD STORAGE}
	\textbf{Cloud storage:} is a method of storing, retrieving and sharing data for matters of Simplicity, reliability and scalability.
	
	 Some of the \textbf{Considerations for cloud storage include:} Durability, Availabity, and Security and there are three \textbf{types of cloud data storage,} and each offers their own advantages and have their own use cases:
	
	
	\textbf{Object Storage:}used for Applications development, solutions like Amazon Simple Storage Service (S3)[1] are ideal for building modern applications from scratch that require scale and flexibility, it also imports existing data stores for analytics, backup, or archive.
	
	
	\textbf{File Storage:} used to access shared files supported with a Network Attached Storage (NAS) server. solutions like Amazon Elastic File System (EFS) are ideal for use cases like large content repositories, development environments, media stores, or user home directories.
	
	
	\textbf{Block Storage:} solutions like Amazon Elastic Block Store (EBS) offer the ultra-low latency required for high performance workloads unlike databases or ERP systems often require dedicated, low latency storage for each host. This is analogous to direct-attached storage (DAS) or a Storage Area Network (SAN).
	
	
	\subsection*{Cloud storage benefits include} Total cost of ownership, Time to development, Information management, Allows backup and recovery unlike physical storage that requires maintenance costs, and information management costs without backup services.
	\subsection*{Cloud storage challenges include} Someone else is looking after your data, Cyber-attacks, Insider threats, Government intrusion, Legal reliability, Lack of standardization, Lack of support.
	
	
	\cite{2}\textbf{Unlike physical storage with assured confidentiality, free from cyber-attacks and government intrusion.}
	
    \textquotedblleft{Spotify uses Google Cloud Storage for storing and serving music. Using Regional storage allowed us to run audio transcoding in Google Compute Engine close to production storage. Google also offers great networking with open and explicit peering setup, as well as interconnect partnerships with all of our CDN providers.}
   \textquotedblright
   
   \cite{3}google cloud storage services offers, a single API for all storage classes and Storage solutions for any workload this makes the best option among all other cloud storage service providing business in terms of cost but amazon provides the best quality cloud services as per the current ratings.
   
	
\newpage
\bibliographystyle{plain}
\bibliography{document}	

	
	

\end{document}

